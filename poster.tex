\documentclass[final]{beamer}

\usepackage[size=custom,width=84.1,height=118.9,scale=1]{beamerposter} % Define paper size (A0 portrait, might need to adjust the scale)
\usepackage{poster}

% ====================
% Lengths
% ====================

% 2 columns

% If you have N columns, choose \sepwidth and \colwidth such that
% (N+1)*\sepwidth + N*\colwidth = \paperwidth
\newlength{\sepwidth}
\newlength{\colwidth}
\setlength{\sepwidth}{0.033\paperwidth}
\setlength{\colwidth}{0.45\paperwidth}

\newcommand{\separatorcolumn}{\begin{column}{\sepwidth}\end{column}}

% ====================
% Title
% ====================

\title{A nice title}

\author{Thomson\inst{1} \and Thompson\inst{2}}

\institute[shortinst]{\inst{1} Some Institute \samelineand \inst{2} Another Institute}

% ====================
% Footer (optional)
% ====================

% \footercontent{
%   \href{https://www.example.com}{https://www.example.com} \hfill
%   ABC Conference 2025, New York --- XYZ-1234 \hfill
%   \href{mailto:alyssa.p.hacker@example.com}{alyssa.p.hacker@example.com}}
% (can be left out to remove footer)

% ====================
% Logo (optional)
% ====================

% use this to include logos on the left and/or right side of the header:
% \logoright{\includegraphics[height=7cm]{logo1.pdf}}
% \logoleft{\includegraphics[height=7cm]{logo2.pdf}}

% Otherwise put logos in a block at the end

% ====================
% Body
% ====================

\begin{document}

\begin{frame}[t]
\begin{columns}[t]

%%%%%%%%% COLUMN 1
\separatorcolumn
\begin{column}{\colwidth}

\begin{block}{Introduction}

\begin{itemize}
    \item Tempore quo primis auspiciis in mundanum fulgorem surgeret victura dum erunt homines Roma, ut augeretur sublimibus incrementis, foedere pacis aeternae
	\item Virtus convenit atque Fortuna plerumque dissidentes, quarum si altera defuisset, ad perfectam non venerat summitatem.
\end{itemize}

\end{block}
  
\begin{alertblock}{Objectives}

Altera sententia est, quae definit amicitiam paribus officiis ac voluntatibus.
Hoc quidem est nimis exigue et exiliter ad calculos vocare amicitiam, ut par sit ratio acceptorum et datorum.
Divitior mihi et affluentior videtur esse vera amicitia nec observare restricte, ne plus reddat quam acceperit; neque enim verendum est, ne quid excidat, aut ne quid in terram defluat, aut ne plus aequo quid in amicitiam congeratur.

\end{alertblock}

\begin{block}{Methods}

\cite{Shmueli2010}

\end{block}

\end{column}

%%%%%%%%% COLUMN 2
\separatorcolumn
\begin{column}{\colwidth}

\begin{block}{Results}

Intellectum est enim mihi quidem in multis, et maxime in me ipso, sed paulo ante in omnibus, cum M. Marcellum senatui reique publicae concessisti, commemoratis praesertim offensionibus, te auctoritatem huius ordinis dignitatemque rei publicae tuis vel doloribus vel suspicionibus anteferre.
Ille quidem fructum omnis ante actae vitae hodierno die maximum cepit, cum summo consensu senatus, tum iudicio tuo gravissimo et maximo.
Ex quo profecto intellegis quanta in dato beneficio sit laus, cum in accepto sit tanta gloria.

\end{block}

\begin{block}{Conclusion}
\large % Larger font

\begin{itemize}
    \item Quaestione igitur per multiplices dilatata fortunas cum ambigerentur quaedam, non nulla levius actitata constaret,
	\item post multorum clades Apollinares ambo pater et filius in exilium acti cum ad locum Crateras nomine pervenissent,
	\item villam scilicet suam quae ab Antiochia vicensimo et quarto disiungitur lapide, ut mandatum est, fractis cruribus occiduntur.
\end{itemize}

\end{block}

\begin{block}{Acknowledgements}

{
\footnotesize % Smaller font
This project was supported by ...
Our thanks to ...
}

\end{block}

\begin{block}{References}
\printbibliography % Fontsize is controlled in poster.sty
\end{block}

\begin{block} % Logo

% ideally put that in the footline of beamer template (with line like for heading?)
\begin{table}[]
    \centering
    \begin{tabular}{ccc}
        %  \includegraphics[height=3cm]{img/Logo1} &
        %  \hspace{2cm} &
        %  \includegraphics[height=3cm]{img/Logo2} \\
    \end{tabular}
\end{table}

\end{block}

\end{column}

\separatorcolumn
\end{columns}

\end{frame}

\end{document}
